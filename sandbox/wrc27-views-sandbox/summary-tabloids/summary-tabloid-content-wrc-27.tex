\newcommand{\AIaTitle}{AI-1.1: Aeronautical/Marine Earth Stations in Motion around 50 GHz}
\newcommand{\AIaFigure}{/users/livesey/corf/njl-corf/sandbox/wrc27-views-sandbox/specific-ai-plots-no-legend/SpecificAI-WRC-27 AI-1.1.pdf}
\newcommand{\AIaText}{Consider the technical and operational conditions for the use of the frequency bands \mbox{47.2-50.2 GHz} and \mbox{50.4-51.4 GHz} (Earth-to-space), or parts thereof, by aeronautical and maritime earth stations in motion communicating with space stations in the fixed-satellite service and develop regulatory measures, as appropriate, to facilitate the use of the frequency bands \mbox{47.2-50.2 GHz} and \mbox{50.4-51.4 GHz} (Earth-to-space), or parts thereof, by aeronautical and maritime earth stations in motion communicating with geostationary space stations and non-geostationary space stations in the fixed-satellite service, in accordance with Resolution 176 (Rev.WRC-23)}
\newcommand{\AIbTitle}{AI-1.2: Fixed satellite uplinks in \mbox{13.75--14\,GHz}}
\newcommand{\AIbFigure}{/users/livesey/corf/njl-corf/sandbox/wrc27-views-sandbox/specific-ai-plots-no-legend/SpecificAI-WRC-27 AI-1.2.pdf}
\newcommand{\AIbText}{Consider possible revisions of sharing conditions in the frequency band \mbox{13.75-14 GHz} to allow the use of uplink fixed-satellite service earth stations with smaller antenna sizes, in accordance with Resolution 129 (WRC-23)}
\newcommand{\AIcTitle}{AI-1.3: Fixed uplinks to non-geostationary satellites at \mbox{51.4--52.4\,GHz}}
\newcommand{\AIcFigure}{/users/livesey/corf/njl-corf/sandbox/wrc27-views-sandbox/specific-ai-plots-no-legend/SpecificAI-WRC-27 AI-1.3.pdf}
\newcommand{\AIcText}{Consider studies relating to the use of the frequency band \mbox{51.4-52.4 GHz} to enable use by gateway earth stations transmitting to non-geostationary-satellite orbit systems in the fixed-satellite service (Earth-to-space), in accordance with Resolution 130 (WRC-23);}
\newcommand{\AIdTitle}{AI-1.4: Fixed satellite downlinks in \mbox{17.3--17.8\,GHz}}
\newcommand{\AIdFigure}{/users/livesey/corf/njl-corf/sandbox/wrc27-views-sandbox/specific-ai-plots-no-legend/SpecificAI-WRC-27 AI-1.4.pdf}
\newcommand{\AIdText}{Consider a possible new primary allocation to the fixed-satellite service (space-to-Earth) in the frequency band \mbox{17.3-17.7 GHz} and a possible new primary allocation to the broadcasting-satellite service (space-to-Earth) in the frequency band \mbox{17.3-17.8 GHz} in Region 3, while ensuring the protection of existing primary allocations in the same and adjacent frequency bands, and to consider equivalent power flux-density limits to be applied in Regions 1 and 3 to non-geostationary-satellite systems in the fixed-satellite service (space-to-Earth) in the frequency band \mbox{17.3-17.7 GHz}, in accordance with Resolution 726 (WRC-23)}
\newcommand{\AIeTitle}{AI-1.5: Limiting unauthorized non-geostationary ground station operations}
\newcommand{\AIeFigure}{/users/livesey/corf/njl-corf/sandbox/wrc27-views-sandbox/specific-ai-plots-no-legend/SpecificAI-WRC-27 AI-1.5.pdf}
\newcommand{\AIeText}{Consider regulatory measures, and implementability thereof, to limit the unauthorized operations of non-geostationary-satellite orbit earth stations in the fixed-satellite and mobile-satellite services and associated issues related to the service area of non-geostationary-satellite orbit satellite systems in the fixed-satellite and mobile-satellite services, in accordance with Resolution 14 (WRC-23)}
\newcommand{\AIfTitle}{AI-1.6: Equitable access to fixed satellite service in selected bands}
\newcommand{\AIfFigure}{/users/livesey/corf/njl-corf/sandbox/wrc27-views-sandbox/specific-ai-plots-no-legend/SpecificAI-WRC-27 AI-1.6.pdf}
\newcommand{\AIfText}{Consider technical and regulatory measures for fixed-satellite service satellite networks/systems in the frequency bands \mbox{37.5-42.5 GHz} (space-to-Earth), \mbox{42.5-43.5 GHz} (Earth-to-space), \mbox{47.2-50.2 GHz} (Earth-to-space) and \mbox{50.4-51.4 GHz} (Earth-to-space) for equitable access to these frequency bands, in accordance with Resolution 131 (WRC-23)}
\newcommand{\AIgTitle}{AI-1.7: Allocations to International Mobile Telecommunications}
\newcommand{\AIgFigure}{/users/livesey/corf/njl-corf/sandbox/wrc27-views-sandbox/specific-ai-plots-no-legend/SpecificAI-WRC-27 AI-1.7.pdf}
\newcommand{\AIgText}{Consider studies on sharing and compatibility and develop technical conditions for the use of International Mobile Telecommunications (IMT) in the frequency bands \mbox{4 400-4 800 MHz}, \mbox{7 125-8 400 MHz} (or parts thereof), and \mbox{14.8-15.35 GHz} taking into account existing primary services operating in these, and adjacent, frequency bands, in accordance with Resolution 256 (WRC-23)}
\newcommand{\AIhTitle}{AI-1.8: Radiolocation from \mbox{231.5--700\,GHz}}
\newcommand{\AIhFigure}{/users/livesey/corf/njl-corf/sandbox/wrc27-views-sandbox/specific-ai-plots-no-legend/SpecificAI-WRC-27 AI-1.8.pdf}
\newcommand{\AIhText}{Consider possible additional spectrum allocations to the radiolocation service on a primary basis in the frequency range \mbox{231.5-275 GHz} and possible new identifications for radiolocation service applications in the frequency bands within the frequency range \mbox{275-700 GHz} for millimetric and sub-millimetric wave imaging systems, in accordance with Resolution 663 (Rev.WRC-23)}
\newcommand{\AIiTitle}{AI-1.9: Updating Appendix 26 for modernization of aeronautical HF spectrum}
\newcommand{\AIiFigure}{/users/livesey/corf/njl-corf/sandbox/wrc27-views-sandbox/specific-ai-plots-no-legend/SpecificAI-WRC-27 AI-1.9.pdf}
\newcommand{\AIiText}{Consider appropriate regulatory actions to update Appendix 26 to the Radio Regulations in support of aeronautical mobile (OR) high frequency modernization, in accordance with Resolution 411 (WRC-23);}
\newcommand{\AIjTitle}{AI-1.10: Power limits for satellites in 71–76 and \mbox{81--86\,GHz}}
\newcommand{\AIjFigure}{/users/livesey/corf/njl-corf/sandbox/wrc27-views-sandbox/specific-ai-plots-no-legend/SpecificAI-WRC-27 AI-1.10.pdf}
\newcommand{\AIjText}{Consider developing power flux-density and equivalent isotropically radiated power limits for inclusion in Article 21 of the Radio Regulations for the fixed-satellite, mobile-satellite and broadcasting-satellite services to protect the fixed and mobile services in the frequency bands \mbox{71-76 GHz} and \mbox{81-86 GHz}, in accordance with Resolution 775 (Rev.WRC-23)}
\newcommand{\AIkTitle}{AI-1.11: Space-to-space transmissions in \mbox{1.5--2.5\,GHz}}
\newcommand{\AIkFigure}{/users/livesey/corf/njl-corf/sandbox/wrc27-views-sandbox/specific-ai-plots-no-legend/SpecificAI-WRC-27 AI-1.11.pdf}
\newcommand{\AIkText}{Consider the technical and operational issues, and regulatory provisions, for space-to-space links among non-geostationary and geostationary satellites in the frequency bands \mbox{1 518-1 544 MHz}, \mbox{1 545-1 559 MHz}, \mbox{1 610-1 645.5 MHz}, \mbox{1 646.5-1 660 MHz}, \mbox{1 670-1 675 MHz} and \mbox{2 483.5-2 500 MHz} allocated to the mobile-satellite service, in accordance with Resolution 249 (Rev.WRC-23)}
\newcommand{\AIlTitle}{AI-1.12: Mobile satellite services from \mbox{1.4--2.025\,GHz}}
\newcommand{\AIlFigure}{/users/livesey/corf/njl-corf/sandbox/wrc27-views-sandbox/specific-ai-plots-no-legend/SpecificAI-WRC-27 AI-1.12.pdf}
\newcommand{\AIlText}{Consider, based on the results of studies, possible allocations to the mobile-satellite service and possible regulatory actions in the frequency bands \mbox{1 427-1 432 MHz} (space-to-Earth), \mbox{1 645.5-1 646.5 MHz} (space-to-Earth) (Earth-to-space), \mbox{1 880-1 920 MHz} (space-to-Earth) (Earth-to-space) and \mbox{2 010-2 025 MHz} (space-to-Earth) (Earth-to-space) required for the future development of low-data-rate non-geostationary mobile-satellite systems, in accordance with Resolution 252 (WRC-23)}
\newcommand{\AImTitle}{AI-1.13: Direct satellite to cellphone communications in 694 MHz to 2.7 GHz}
\newcommand{\AImFigure}{/users/livesey/corf/njl-corf/sandbox/wrc27-views-sandbox/specific-ai-plots-no-legend/SpecificAI-WRC-27 AI-1.13.pdf}
\newcommand{\AImText}{Consider studies on possible new allocations to the mobile-satellite service for direct connectivity between space stations and International Mobile Telecommunications (IMT) user equipment to complement terrestrial IMT network coverage, in accordance with Resolution 253 (WRC-23);}
\newcommand{\AInTitle}{AI-1.14: Mobile satellite service in \mbox{2.0--2.2\,GHz}}
\newcommand{\AInFigure}{/users/livesey/corf/njl-corf/sandbox/wrc27-views-sandbox/specific-ai-plots-no-legend/SpecificAI-WRC-27 AI-1.14.pdf}
\newcommand{\AInText}{Consider possible additional allocations to the mobile-satellite service, in accordance with Resolution 254 (WRC-23)}
\newcommand{\AIoTitle}{AI-1.15: Communications in the Lunar environment}
\newcommand{\AIoFigure}{/users/livesey/corf/njl-corf/sandbox/wrc27-views-sandbox/specific-ai-plots-no-legend/SpecificAI-WRC-27 AI-1.15.pdf}
\newcommand{\AIoText}{Consider studies on frequency-related matters, including possible new or modified space research service (space-to-space) allocations, for future development of communications on the lunar surface and between lunar orbit and the lunar surface, in accordance with Resolution 680 (WRC-23)}
\newcommand{\AIpTitle}{AI-1.16: Protection of Radio Quiet Zones}
\newcommand{\AIpFigure}{/users/livesey/corf/njl-corf/sandbox/wrc27-views-sandbox/specific-ai-plots-no-legend/SpecificAI-WRC-27 AI-1.16.pdf}
\newcommand{\AIpText}{Consider studies on the technical and regulatory provisions necessary to protect radio astronomy operating in specific Radio Quiet Zones and, in frequency bands allocated to the radio astronomy service on a primary basis globally, from aggregate radio-frequency interference caused by non-geostationary-satellite orbit systems, in accordance with Resolution 681 (WRC-23)}
\newcommand{\AIqTitle}{AI-1.17: Allocations for receive-only space weather sensors}
\newcommand{\AIqFigure}{/users/livesey/corf/njl-corf/sandbox/wrc27-views-sandbox/specific-ai-plots-no-legend/SpecificAI-WRC-27 AI-1.17.pdf}
\newcommand{\AIqText}{Consider regulatory provisions for receive-only space weather sensors and their protection in the Radio Regulations, taking into account the results of ITU Radiocommunication Sector studies, in accordance with Resolution 682 (WRC-23)}
\newcommand{\AIrTitle}{AI-1.18: Protection of EESS (passive) and RAS in bands above 76 GHz}
\newcommand{\AIrFigure}{/users/livesey/corf/njl-corf/sandbox/wrc27-views-sandbox/specific-ai-plots-no-legend/SpecificAI-WRC-27 AI-1.18.pdf}
\newcommand{\AIrText}{Consider, based on the results of ITU Radiocommunication Sector studies, possible regulatory measures regarding the protection of the Earth exploration-satellite service (passive) and the radio astronomy service in certain frequency bands above 76 GHz from unwanted emissions of active services, in accordance with Resolution 712 (WRC-23)}
\newcommand{\AIsTitle}{AI-1.19: Allocations to EESS in 4.2–4.4 and \mbox{8.4--8.5\,GHz}}
\newcommand{\AIsFigure}{/users/livesey/corf/njl-corf/sandbox/wrc27-views-sandbox/specific-ai-plots-no-legend/SpecificAI-WRC-27 AI-1.19.pdf}
\newcommand{\AIsText}{Consider possible primary allocations in all Regions to the Earth exploration-satellite service (passive) in the frequency bands \mbox{4 200-4 400 MHz} and \mbox{8 400-8 500 MHz}, in accordance with Resolution 674 (WRC-23)}
